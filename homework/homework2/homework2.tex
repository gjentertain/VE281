\documentclass{article}
\usepackage{enumerate}
\usepackage{amsmath}
\usepackage{amssymb}
\usepackage{graphicx}
\usepackage{subfigure}
\usepackage{geometry}
\usepackage{caption}
\usepackage{indentfirst}
\usepackage{array}
%\renewcommand\arraystretch{2}

\usepackage{algorithm}  
\usepackage{algorithmicx}  
\usepackage{algpseudocode}
\renewcommand{\algorithmicrequire}{\textbf{Input:}}  
\renewcommand{\algorithmicensure}{\textbf{Output:}}  

\geometry{left=3.0cm,right=3.0cm,top=3.0cm,bottom=4.0cm}
\renewcommand{\thesection}{Ex. \arabic{section}}
\title{VE281 Writing Assignment Two}
\author{Liu Yihao 515370910207}
\date{}

\begin{document}
\maketitle

\section{}
\begin{center}
\begin{tabular}{|c|c|c|c|c|c|c|c|c|c|c|c|c|}
\hline
step & pivot & 0 & 1 & 2 & 3 & 4 & 5 & 6 & 7 & 8 & 9 & comment \\\hline
1 & 5 & 5 & 7 & 4 & 1 & 8 & 9 & 2 & 6 & 3 & 10 & swap arr[0] and pivot \\\hline
1 & 5 & 5 & 3 & 4 & 1 & 2 & 9 & 8 & 6 & 7 & 10 & partition \\\hline
1 & 5 & 2 & 3 & 4 & 1 & 5 & 9 & 8 & 6 & 7 & 10 & swap arr[0] and partition pos \\\hline
2 & 2 & 2 & 3 & 4 & 1 &&&&&&& swap arr[0] and pivot \\\hline
2 & 2 & 2 & 1 & 4 & 3 &&&&&&& partition \\\hline
2 & 2 & 1 & 2 & 4 & 3 &&&&&&& swap arr[0] and partition pos \\\hline
3 &   &   &   & 3 & 4 &&&&&&& insertion sort arr[2],arr[3] \\\hline
4 & 8 &&&&&& 8 & 9 & 6 & 7 & 10 & swap arr[5] and pivot\\\hline
4 & 8 &&&&&& 8 & 7 & 6 & 9 & 10 & partition \\\hline
4 & 8 &&&&&& 6 & 7 & 8 & 9 & 10 & swap arr[5] and partition pos \\\hline
5 &   &&&&&& 6 & 7 &   &   &    & insertion sort arr[5],arr[6] \\\hline
6 &   &&&&&&   &   &   & 9 & 10 & insertion sort arr[8],arr[9] \\\hline
\end{tabular}
\end{center}

\section{}
\begin{center}
\begin{tabular}{|c|m{2em}<{\centering}|m{2em}<{\centering}|m{2em}<{\centering}|m{2em}<{\centering}|m{2em}<{\centering}|m{2em}<{\centering}|m{2em}<{\centering}|m{2em}<{\centering}|m{2em}<{\centering}|m{2em}<{\centering}|}
\hline
step & 0 & 1 & 2 & 3 & 4 & 5 & 6 & 7 & 8 & 9 \\\hline
1 & & & 032 632 & 943 & & & 446 526 & & 538 738 & 189 479 379 \\\hline 
2 & & & 526 & 032 632 538 738 & 943 446 & & & 479 379 & 189 & \\\hline
3 & 032 & 189 & & 379 & 446 479 & 526 538 & 632 & 738 & & 943\\\hline
\end{tabular}
\end{center}

\section{}
Similar to the proof in the slides, when $k=n/7$, approx at least $5/7/2=5/14$ is smaller than $x_{k/2}$, and at least $5/7/2=5/14$ is larger than $x_{k/2}$, so we can find the recurrence relationship
$$T(n)=cn+T\left(\frac{n}{7}\right)+T\left(\frac{9n}{14}\right)$$

Suppose there exists a positive constant $c$ such that 
$$T(1)\leqslant c$$
$$T(n)\leqslant cn+T\left(\frac{n}{7}\right)+T\left(\frac{9n}{14}\right)$$

Then $$T(n)\leqslant 14cn$$

For the base case, obviously $T(1)\leqslant cn\leqslant 14cn$.

For the inductive step,
$$T(n)\leqslant cn+T\left(\frac{n}{7}\right)+T\left(\frac{9n}{14}\right)\leqslant cn+2cn+9cn \leqslant 14cn$$

So the runtime of this new algorithm is still $O(n)$.

\section{}
\begin{enumerate}[(a)]
\item \ 
\begin{center}
\begin{tabular}{|m{2em}<{\centering}|m{2em}<{\centering}|m{2em}<{\centering}|m{2em}<{\centering}|m{2em}<{\centering}|m{2em}<{\centering}|m{2em}<{\centering}|m{2em}<{\centering}|m{2em}<{\centering}|m{2em}<{\centering}|}
\hline
0 & 1 & 2 & 3 & 4 & 5 & 6 & 7 & 8 & 9 \\\hline
& 4371 & & 6173 1323 & 4344 & & & & & 1989 9679 4199 \\\hline
\end{tabular}
\end{center}
\item \ 
\begin{center}
\begin{tabular}{|m{2em}<{\centering}|m{2em}<{\centering}|m{2em}<{\centering}|m{2em}<{\centering}|m{2em}<{\centering}|m{2em}<{\centering}|m{2em}<{\centering}|m{2em}<{\centering}|m{2em}<{\centering}|m{2em}<{\centering}|}
\hline
0 & 1 & 2 & 3 & 4 & 5 & 6 & 7 & 8 & 9 \\\hline
9679 & 4371 & 1989 & 1323 & 6173 & 4344 & & & & 4199 \\\hline
\end{tabular}
\end{center}
\item \ 
\begin{center}
\begin{tabular}{|m{2em}<{\centering}|m{2em}<{\centering}|m{2em}<{\centering}|m{2em}<{\centering}|m{2em}<{\centering}|m{2em}<{\centering}|m{2em}<{\centering}|m{2em}<{\centering}|m{2em}<{\centering}|m{2em}<{\centering}|}
\hline
0 & 1 & 2 & 3 & 4 & 5 & 6 & 7 & 8 & 9 \\\hline
9679 & 4371 & & 1323 & 6173 & 4344 &&& 1989 & 4199 \\\hline
\end{tabular}
\end{center}
\item \ 
\begin{center}
\begin{tabular}{|m{2em}<{\centering}|m{2em}<{\centering}|m{2em}<{\centering}|m{2em}<{\centering}|m{2em}<{\centering}|m{2em}<{\centering}|m{2em}<{\centering}|m{2em}<{\centering}|m{2em}<{\centering}|m{2em}<{\centering}|}
\hline
0 & 1 & 2 & 3 & 4 & 5 & 6 & 7 & 8 & 9 \\\hline
 & 4371 & & 1323 & 6173 & 9679 & & 4344 & & 4199 \\\hline
\end{tabular}
\end{center}
Then we can find that $(7-1989)\equiv 6\mod 7$, however, the odd slots are not empty, so this element can't be inserted.
\end{enumerate}

\section{}
\begin{enumerate}[(a)]
\item \ 
\begin{center}
\begin{tabular}{|m{2em}<{\centering}|m{2em}<{\centering}|m{2em}<{\centering}|m{2em}<{\centering}|m{2em}<{\centering}|m{2em}<{\centering}|m{2em}<{\centering}|m{2em}<{\centering}|m{2em}<{\centering}|m{2em}<{\centering}|}
\hline
0 & 1 & 2 & 3 & 4 & 5 & 6 & 7 & 8 & 9 \\\hline
4199 & 4371 &&&&&&& 9679 & \\\hline
10 & 11 & 12 & 13 & 14 & 15 & 16 & 17 & 18 & \\\hline
& & 4344 1323 & 1989 & & & & 6173 & & \\\hline
\end{tabular}
\end{center}
\item \ 
\begin{center}
\begin{tabular}{|m{2em}<{\centering}|m{2em}<{\centering}|m{2em}<{\centering}|m{2em}<{\centering}|m{2em}<{\centering}|m{2em}<{\centering}|m{2em}<{\centering}|m{2em}<{\centering}|m{2em}<{\centering}|m{2em}<{\centering}|}
\hline
0 & 1 & 2 & 3 & 4 & 5 & 6 & 7 & 8 & 9 \\\hline
4199 & 4371 &&&&&&& 9679 & \\\hline
10 & 11 & 12 & 13 & 14 & 15 & 16 & 17 & 18 & \\\hline
& & 1323 & 1989 & 4344 & & & 6173 & & \\\hline
\end{tabular}
\end{center}
\item \ 
\begin{center}
\begin{tabular}{|m{2em}<{\centering}|m{2em}<{\centering}|m{2em}<{\centering}|m{2em}<{\centering}|m{2em}<{\centering}|m{2em}<{\centering}|m{2em}<{\centering}|m{2em}<{\centering}|m{2em}<{\centering}|m{2em}<{\centering}|}
\hline
0 & 1 & 2 & 3 & 4 & 5 & 6 & 7 & 8 & 9 \\\hline
4199 & 4371 &&&&&&& 9679 & \\\hline
10 & 11 & 12 & 13 & 14 & 15 & 16 & 17 & 18 & \\\hline
& & 1323 & 4344 & 1989 & & & 6173 & & \\\hline
\end{tabular}
\end{center}
\item\ 
\begin{center}
\begin{tabular}{|m{2em}<{\centering}|m{2em}<{\centering}|m{2em}<{\centering}|m{2em}<{\centering}|m{2em}<{\centering}|m{2em}<{\centering}|m{2em}<{\centering}|m{2em}<{\centering}|m{2em}<{\centering}|m{2em}<{\centering}|}
\hline
0 & 1 & 2 & 3 & 4 & 5 & 6 & 7 & 8 & 9 \\\hline
4199 & 4371 && &&&&& 9679 & \\\hline
10 & 11 & 12 & 13 & 14 & 15 & 16 & 17 & 18 & \\\hline
& & 1323 & 1989 & & 4344 & & 6173 & & \\\hline
\end{tabular}
\end{center}
\end{enumerate}

\end{document}
