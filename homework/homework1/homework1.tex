\documentclass{article}
\usepackage{enumerate}
\usepackage{amsmath}
\usepackage{amssymb}
\usepackage{graphicx}
\usepackage{subfigure}
\usepackage{geometry}
\usepackage{caption}
\usepackage{indentfirst}

\usepackage{algorithm}  
\usepackage{algorithmicx}  
\usepackage{algpseudocode}
\renewcommand{\algorithmicrequire}{\textbf{Input:}}  
\renewcommand{\algorithmicensure}{\textbf{Output:}}  

\geometry{left=3.0cm,right=3.0cm,top=3.0cm,bottom=4.0cm}
\renewcommand{\thesection}{Ex. \arabic{section}}
\title{VE281 Writing Assignment One}
\author{Liu Yihao 515370910207}
\date{}

\begin{document}
\maketitle

\section{}
The strategy is:
\begin{algorithm}[H]
    \begin{algorithmic}
        \Require maximum integer $N$
        \Ensure correct integer $x$
        \State $begin\gets1$, $end\gets N$
        \While {True}
        		\State $x\gets\lfloor(begin+end)/2\rfloor$
        		\State $result\gets$ make a guess with $x$
        		\If {$result=$ ``equal to''}{ \textbf{break}}
        		\ElsIf {$result=$ ``less then''}{ $end\gets x-1$}
        		\Else { $begin\gets x+1$}
        		\EndIf
        \EndWhile
    \end{algorithmic}  
\end{algorithm}

In the worst case, I never directly guess the correct number in the process of the strategy. After a guess of $N$, the next range of numbers become $(N-1)/2$. When $N=2^m-1$, the last guess will contain $2^2-1=3$ numbers, and I'll get the correct one after the guesses. So the number of guess in the worst case is $m-1$. \\

In the average case, if there are $M$ numbers left, I've got the probability of $1/M$ to guess the number directly. Let the probability of guessing $n$ times be $P_n$,
$$P_1=\frac{1}{2^m-1}$$ 
$$P_n=\frac{1-P_{n-1}}{2^{m-n+1}-1}=\frac{2^{n-1}}{2^{m}-1}$$ 

Then we can get the equation
$$T_m =\sum_{n=1}^{m-1}nP_n=\frac{1\cdot2^0+2\cdot2^1+\cdots+n\cdot2^{m-2}}{2^{m}-1}=\frac{(m-2)2^{m-1}+1}{2^m-1}$$

\section{}
In the best situation, if $n=2^m$, only $cm$ steps ($c$ is a constant) is necessary. So $f(n)=\log n$.

\section{}
$$\lim_{n\to\infty}\frac{n^{100}}{1.001^n}=\lim_{n\to\infty}\frac{100n^{99}}{1.001^n\ln1.001}=\cdots=\lim_{n\to\infty}\frac{100!}{1.001^n\ln^{100}1.001}=0$$

So $n^{100}=O(1.001^n)$ is false.

\section{}
Since $f_1(n)=\Theta(g_1(n))$, $f_2(n)=\Theta(g_2(n))$, we know
$$c_1g_1(n)\leqslant f_1(n)\leqslant c_2g_1(n),n\geqslant n_1$$
$$c_3g_2(n)\leqslant f_2(n)\leqslant c_4g_2(n),n\geqslant n_2$$

Then
$$c_1g_1(n)+c_3g_2(n)\leqslant f_1(n)+f_2(n)\leqslant c_2g_1(n)+c_4g_2(n),n\geqslant \max\{n_1,n_2\}$$

Now we should find $h(n)$ so that
$$\lim_{n\to\infty}\frac{c_1g_1(n)+c_3g_2(n)}{h(n)}=C_1$$
$$\lim_{n\to\infty}\frac{c_2g_2(n)+c_4g_2(n)}{h(n)}=C_2$$

So $$h(n)=\max\{g_1(n),g_2(n)\}$$

\section{}
The outer loop will be called $\lfloor1+\log_an\rfloor$ times, in $k^{th}$ loop, $i=a^k$, where $k\in\Big[0,\lfloor\log_an\rfloor\Big]\cap Z$.

The inner loop will be called $ib$ times in each loop.

$$T=b\sum_{k=0}^{\lfloor\log_an\rfloor}a^k=\frac{b\left(1-a^{\lfloor\log_an\rfloor}\right)}{1-a}$$

\section{}
\begin{enumerate}[(a)]
\item
$$T_1(n)=\lceil n/2\rceil$$
$$T_1(n)=\lceil n/20\rceil$$
$$T_3(n)=\lceil 1+\log_2n\rceil$$
\item
$$C_1(n)=4n+2\lceil n/2\rceil$$
$$C_2(n)=10+4n+\lceil n/20\rceil$$
$$C_3(n)=40+4n+2\lceil 1+\log_2n\rceil$$
\item
Method 1 is the cheapest, it costs \$40.
\item
Method 2 is the cheapest, it costs \$529.
\item
Method 3 is the cheapest, it costs \$8256.
\end{enumerate}


\end{document}
